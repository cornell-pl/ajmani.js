\documentclass[a4paper,10pt]{article}
\usepackage[utf8]{inputenc}
\usepackage{fullpage}
\usepackage{amsmath}
\usepackage{amsthm}
\usepackage{amssymb}
\usepackage{color}
\usepackage{mathpartir}

\newtheorem{definition}{Definition}

\newcommand{\finish}[1]{#1}
\newcommand{\comment}[1]{\finish{\textbf{\textcolor{red}{#1}}}}
\newcommand{\jnf}[1]{\finish{\textbf{\textcolor{blue}{[#1---JNF]}}}}
\newcommand{\raghu}[1]{\finish{\textbf{\textcolor{blue}{[#1---Raghu]}}}}

\newcommand{\defeq}{\ensuremath{\stackrel{\Delta}{=}}}
\newcommand{\Unit}{\ensuremath{\mathbf{Unit}}}

%opening
\title{}
\author{}

\begin{document}

\maketitle

\section{Category-theoretic Model of lenses}
\begin{itemize}
 \item Categories are types equipped with a monoid of updates. Objects of the category are values, and morphisms between objects are updates corresponding to the monoid action. Thus, the pair of a source object and update corresponds to a single morphism.
 \item Asymmetric Lenses are opfibrations (functors) between categories.
 \item A correspondence between two categories is specified by its ``l-diagram''. In the simple case (no state), the l-diagram is just $X \to X \cap Y \leftarrow Y$.
 \item Every l-diagram has a limit, defined the \emph{ideal}. 
 \item A symmetric I-lens between categories X and Y consists of asymmetric lenses between X and I, and Y and I; where I is the ideal of the l-diagram of X and Y.
\end{itemize}


\section{Ideal Lenses}
\begin{definition}
 An ideal lens $l : X \leftrightarrow Y$ consists of 
 \begin{enumerate}
  \item an ideal $I$
  \item Asymmetric lenses $l : I \leftrightarrow X$ and $l' : I \leftrightarrow Y$
 \end{enumerate}

\end{definition}

\section{I-lenses in a local setting}
  Thinking of the ideal $I$ as isomorphic to $X \times Y \times C$, where $C$ is the type of lens state, it may be superior to have the ideal materialized, instead of $X$, $Y$, and $C$. 

\section{I-lenses in a distributed setting}

\section{Questions}
\begin{enumerate}
 \item Labeled morphisms. $dX \times I \to dI, dI \to dY$ versus $lX \to lI, lI \times I \to lY$.
 \item Composing opfibrations. Can we have a lifting $(I \to dI) \to (I' \to dI')$.
 \item The ideal is not something that is intuitive to a lens designer. Instead, the l-diagram can be a more intuitive thing, from which we can perhaps construct the ideal.
\end{enumerate}



\end{document}
