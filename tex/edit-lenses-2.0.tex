\documentclass[a4paper,10pt]{article}
\usepackage[utf8]{inputenc}
\usepackage{color}

\newcommand{\finish}[1]{#1}
\newcommand{\comment}[1]{\finish{\textbf{\textcolor{red}{#1}}}}
\newcommand{\jnf}[1]{\finish{\textbf{\textcolor{blue}{[#1---JNF]}}}}
\newcommand{\raghu}[1]{\finish{\textbf{\textcolor{blue}{[#1---Raghu]}}}}

%opening
\title{}
\author{}

\begin{document}

\maketitle

\section{Edit lenses with modifiable complement, aka M-Edit Lenses}
An edit lens between modules X and Y is $\langle C, init_C, putr, putl \rangle$. \\
The type of $putr$, $dX \times C \to dY \times C$ can be expressed as the isomorphic type $(dX \times C \to dY, dX \to C \to C)$. Thus, we can view the $putr$ function as having two components; one that translates an edit on $X$ into an edit on $Y$ (using $Y$-data from the complement), and another that updates the complement (its $X$-data) with respect to the edit on $X$.

[NB: We can view data as being $X$-data, $Y$-data, shared data, or mixed data. An example of mixed data is a mapping of indices between two lists. The collection of $X$-indices can be considered $X$-data (and similarly for $Y$), but the mapping itself depends on both. This isn't shared data as it isn't data that is present in both $X$ and $Y$ separately.]

For a symmetric lens, the complement represents the data that is not shared. In an edit lens, there is may be no necessity to capture all this data. For example, if we consider a list-mapping edit lens where the updates are list reorderings, the complement could just be $()$. 

Alternatively, let us compare symmetric and edit lenses between $A \times D$ and $A \times E$, where updates include a filter operation. The complement for the symmetric lens would be $D \times E$. 
The complement for the edit lens needs to be $A \times D \times E$; this is because we need to write the transformation $d(A \times D) \to C \to C$, and doing so requires access to the data in $A$. 
Thus, for a sufficiently rich set of updates, the complement oftentimes subsumes the entirety of the data in the two sources.
[This does not arise in Hofmann et. al.'s paper, as the monoids of edits they consider are not rich enough.]

Ideally, the complement update part of the $putr$ transformation would have the type $(dX, X) \to C \to C$. The changes to the $X$-related components of $C$ can be calcluated from the first argument, and the remainder of $C$ can be propagated from the second argument. 

The complement must contain all the mixed data, since it doesn't exist anywhere else. \comment{Big if} If the complement contains all the $X$-data, then the complement update part of $putr$ could have its type replaced with $(dX, Y) \to C \to C$, since the shared data is contained in both $X$ and $Y$. 

\subsection{Definition of M-Edit lens}
TODO

\subsection{Embedding Edit lenses in M-Edit lenses}
TODO

\subsection{Example M-Edit lens}
TODO

\section{Converting an M-Edit lens into a pair of (asymmetric) M-Edit lenses}
In the setting of evolution/migration, only one of the two sources (in addition to the complement) is materialized. In such a scenario, it may be efficient to combine the materialized source with the complement to create a 'super-structure', and use lenses to separately translate between this structure and both sources. 

This may not be viable in alternative settings, particularly in distributed settings, where both sources need to materialized. Alternatively, there may be security concerns that don't allow sharing all the data.

There may not be a one-size-fits-all way to do this for an arbitrary M-Edit lens.

\end{document}
